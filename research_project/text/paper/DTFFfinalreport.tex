\documentclass[12pt]{article}
\usepackage{amsfonts,amsmath,amssymb, listings}

\usepackage[utf8]{inputenc}
\usepackage{graphicx}
\usepackage{multirow}
\usepackage{xcolor}

\usepackage{threeparttable}
\usepackage{blindtext}


\usepackage[a4paper, total={6in, 8in}]{geometry}

\usepackage[numbered,framed]{matlab-prettifier}
%\usepackage{color}

%\setcounter{MaxMatrixCols}{10}

\def\stackunder#1#2{\mathrel{\mathop{#2}\limits_{#1}}}%

\newcommand{\fiid}{\func{i.i.d.}} % use this as in: X_i \stackrel{\fiid}{\sim} \func{Exp} \left( \lambda \right)
\DeclareMathOperator{\Norm}{N}

\newcommand{\platzo}{\vspace*{2cm}}
\newcommand{\platzt}{\vspace*{4cm}}

\newcommand{\findep}{\func{ind}}  % could use indep instead if ind, but 1st book uses ind.

\newcommand{\R}{\ensuremath{{\mathbb R}}}
\newcommand{\N}{\ensuremath{{\mathbb N}}}
\def\QATOP#1#2{{#1 \atop #2}}
\def\QTATOP#1#2{{\textstyle {#1 \atop #2}}}
\def\QDATOP#1#2{{\displaystyle {#1 \atop #2}}}
\voffset=-2.54cm\hoffset=-2.54cm \textheight27cm \textwidth17.0cm \topmargin0.5cm
\oddsidemargin2.00cm \evensidemargin2.00cm \unitlength1cm

\def\func#1{\mathop{\rm #1}}%
\def\dint{\mathop{\displaystyle \int}}%
\newcommand{\Ind}{\ensuremath{{\mathbb I}}} % indicator function
\newcommand{\E}{\ensuremath{{\mathbb E}}} % expected value
\newcommand{\Var}{\ensuremath{{\mathbb V}}} % variance

\setlength{\parindent}{0pt}

\usepackage{float}

\newfloat{Program}{thp}{lop}[section]
\floatname{Program}{Program Listing}

\begin{document}

\pagestyle{empty}

\bigskip

\begin{figure}[htp]
    \centering
    \includegraphics[width=4cm]{uzh logo 2.png}
    \label{fig:UZH}
\end{figure}

\begin{Large}
	\begin{center}
		\textbf{Digital Tools for Finance}
	\end{center}
\end{Large}

\vspace{2cm}

\begin{large}	
	\begin{center}
		\textbf{Insert title - Research BTC} \vspace{0.1cm} \\ {Prof. Igor Pozdeev? } \vspace{2cm} \\ \textbf{Matthias Olieslagers - 22714034}  \\ \textbf{Cameron Storey - 22714034} \\ \textbf{Marc ... - 22714034} \vspace{2cm}
	\end{center}
\end{large}

\tableofcontents

\newpage

\bigskip
\section{Introduction}

\subsection*{subsection 1}


\begin{Program}[!htb]
\begin{lstlisting}[style=Matlab-editor,basicstyle=\mlttfamily\footnotesize]
interesting if we wanna add the code 
\end{lstlisting}
\caption{Question 1 - Part 1}
\label{Question 1 - Part 1}
\end{Program}

\begin{Program}[!htb]
\begin{lstlisting}[style=Matlab-editor,basicstyle=\mlttfamily\footnotesize]
more code 
\end{lstlisting}
\caption{Question 1 - Part 2}
\label{Question 1 - Part 2}
\end{Program}


\begin{center}
\begin{tabular}{||c | c c||} 
 \hline
 Sample size T & Length CI (nonparametric) & Length CI (parametric) \\ [0.5ex] 
 \hline\hline
 250 & 2.5471 & 2.3270 \\ 
 \hline
 500 & 2.1019 & 1.6376 \\
 \hline
 2000 & 1.0723 & 0.8013 \\ [1ex] 
 \hline
\end{tabular}
\end{center}
\vspace{3pt}
\begin{center}
\begin{tabular}{||c | c c||} 
 \hline
 Sample size T & Coverage (nonparametric) & Coverage (parametric) \\ [0.5ex] 
 \hline\hline
 250 & 0.7680 & 0.8680 \\ 
 \hline
 500 & 0.8280 & 0.8760 \\
 \hline
 2000 & 0.8800 & 0.8920 \\ [1ex] 
 \hline
\end{tabular}
\end{center}
\vspace{15pt}

\section{Data}

\begin{Program}[!htb]
\begin{lstlisting}[style=Matlab-editor,basicstyle=\mlttfamily\footnotesize]
dummy
  
\end{lstlisting}
\caption{Question 2 - Part 1}
\label{Question 2 - Part 1}
\end{Program}


1) \textbf{Example 1}: $mu = 0, df = 3, T = 100$     \newline ---     ES simulation = -6.4912, ES numeric int = -6.7485,  \textbf{Deviation of ES's = 0.2573}  \newline
2) \textbf{Example 2}: $mu = -1, df =3, T = 500$ \newline --- ES simulation = -12.8827, ES numeric int = -12.8923 , \textbf{Deviation of ES's = 0.0096} \newline
3) \textbf{Example3}: $mu = -3, df = 6, T = 2000$  \newline--- ES simulation = -15.6922, ES numeric int = -15.6920 , \textbf{Deviation of ES's = 0.0002}\newline\newline

\begin{center}
\begin{tabular}{||c | c | c c||} 
 \hline
 Sample size T & Value $\mu$ & Length CI (nonparametric) & Length CI (parametric) \\ [0.5ex] 
 \hline\hline
 \multirow{4}{4em}{T=100} & $\mu$ = 0 & 5.2666 & 8.9319\\ 
& $\mu$ = -1 & 9.6449 & 9.7008 \\ 
& $\mu$ = -2 & 13.2334 & 19.2469 \\ 
& $\mu$ = -3 & 19.2560 & 43.7532 \\ 
 \hline
 \multirow{4}{4em}{T=500} & $\mu$ = 0 & 3.0341 & 2.5200\\ 
& $\mu$ = -1 & 5.5520 & 3.1778 \\ 
& $\mu$ = -2 & 7.9358 & 4.9203 \\ 
& $\mu$ = -3 & 10.3040 & 7.5688 \\ 
 \hline
 \multirow{4}{4em}{T=2000} & $\mu$ = 0 & 1.6885 & 1.2255\\ 
& $\mu$ = -1 & 2.7962 & 1.5368 \\ 
& $\mu$ = -2 & 4.6726 & 2.4437 \\ 
& $\mu$ = -3 & 5.9613 & 3.6360 \\ 
 \hline
\end{tabular}
\end{center}

\begin{center}
\begin{tabular}{||c | c | c c||} 
 \hline
 Sample size T & Value $\mu$ & Coverage (nonparametric) & Coverage (parametric) \\ [0.5ex] 
 \hline\hline
 \multirow{4}{4em}{T = 100} & $\mu$ = 0 & 0.6800 & 0.8520\\ 
& $\mu$ = -1 & 0.6440 & 0.6080 \\ 
& $\mu$ = -2 & 0.6480 & 0.5800 \\ 
& $\mu$ = -3 & 0.6880 & 0.6240 \\ 
 \hline
 \multirow{4}{4em}{T = 500} & $\mu$ = 0 & 0.7920 & 0.8800\\ 
& $\mu$ = -1 & 0.8560 & 0.1960 \\ 
& $\mu$ = -2 & 0.8040 & 0.1400 \\ 
& $\mu$ = -3 & 0.8160 & 0.2040 \\ 
 \hline
 \multirow{4}{4em}{T=2000} & $\mu$ = 0 & 0.8880 & 0.8880\\ 
& $\mu$ = -1 & 0.8640 & 0.0000 \\ 
& $\mu$ = -2 & 0.8960 & 0.0000 \\ 
& $\mu$ = -3 & 0.8480 & 0.0040 \\ 
 \hline
\end{tabular}
\end{center}

\begin{itemize}
  \item For both bootstrap methods, the length of the CI increases as the non-centrality increases.
  \item For both methods, when the sample size T increases, the length of the CI becomes smaller.
  \item Overall, the length of the CI is smaller for the parametric bootstrap method. Although in this table, this is not the case for T = 100, probably because T is too small. On the next page however (with df = 6) it does give that result also for T = 100 already.\newline
\end{itemize}



\begin{center}
\begin{tabular}{||c | c | c c||} 
 \hline
 Sample size T & Value $\mu$ & Length CI (nonparametric) & Length CI (parametric) \\ [0.5ex] 
 \hline\hline
 \multirow{4}{4em}{T=100} & $\mu$ = 0 & 2.6481 & 2.5196\\ 
& $\mu$ = -1 & 3.6423 & 2.9212 \\ 
& $\mu$ = -2 & 4.9067 & 3.9172 \\ 
& $\mu$ = -3 & 5.9148 & 4.8387 \\ 
 \hline
 \multirow{4}{4em}{T=500} & $\mu$ = 0 & 1.4005 & 1.0989\\ 
& $\mu$ = -1 & 1.9349 & 1.2212 \\ 
& $\mu$ = -2 & 2.5379 & 1.5534 \\ 
& $\mu$ = -3 & 3.2275 & 2.0002 \\ 
 \hline
 \multirow{4}{4em}{T=2000} & $\mu$ = 0 & n/a & n/a\\ 
& $\mu$ = -1 & n/a & n/a \\ 
& $\mu$ = -2 & n/a & n/a \\ 
& $\mu$ = -3 & 1.6814 & 0.9832 \\ 
 \hline
\end{tabular}
\end{center}

\begin{center}
\begin{tabular}{||c | c | c c||} 
 \hline
 Sample size T & Value $\mu$ & Coverage (nonparametric) & Coverage (parametric) \\ [0.5ex] 
 \hline\hline
 \multirow{4}{4em}{T=100} & $\mu$ = 0 & 0.7200 & 0.8360\\ 
& $\mu$ = -1 & 0.7040 & 0.6000 \\ 
& $\mu$ = -2 & 0.7680 & 0.5560 \\ 
& $\mu$ = -3 & 0.6800 & 0.4640 \\ 
 \hline
 \multirow{4}{4em}{T=500} & $\mu$ = 0 & 0.8440 & 0.8720\\ 
& $\mu$ = -1 & 0.8840 & 0.2440 \\ 
& $\mu$ = -2 & 0.8600 & 0.1080 \\ 
& $\mu$ = -3 & 0.8280 & 0.0880 \\ 
 \hline
 \multirow{4}{4em}{T=2000} & $\mu$ = 0 & n/a & n/a\\ 
& $\mu$ = -1 & n/a & n/a \\ 
& $\mu$ = -2 & n/a & n/a \\ 
& $\mu$ = -3 & 0.8920 & 0.0000 \\ 
 \hline
\end{tabular}
\end{center}


\newpage
\section{Methodology}

\begin{Program}[!htb]
\begin{lstlisting}[style=Matlab-editor,basicstyle=\mlttfamily\footnotesize]
more code xd
\end{lstlisting}
\caption{Question 3 }
\label{Question 3 }
\end{Program}

\section{Results}

\newpage
\section{Appendix}

\begin{Program}[!htb]
\begin{lstlisting}[style=Matlab-editor,basicstyle=\mlttfamily\footnotesize]
%Q1 - Function for MLE optimization ---------------------------------------
function MLE = MLE_optimization_Q1(x,initvec)
tol=1e-5;
opts=optimset('Disp ', 'none ' , 'LargeScale ' , 'Off ' , ...
'TolFun ' ,tol , 'TolX ' ,tol , 'Maxiter ' ,200) ;
MLE = fminunc(@(param ) tloglik (param,x) , initvec , opts) ;

function ll = tloglik(param,x)
mu=param ( 1 ) ; c=param ( 2 ) ; v=param ( 3 ) ; 
% i changed the order of the parameters to match the 
% order of the built-in matlab mle output
if v<0.01, v=rand ;  end 
if c<0.01 , c=rand ; end 
K=beta ( v /2 ,0.5)*sqrt( v ) ; z=(x-mu) / c ;
ll = -log(c) -log(K) -(( v+1) /2) * log (1 + (z.^2) / v ); ll = -sum( ll ) ;
\end{lstlisting}
\caption{Q1 - Function for MLE optimization}
\label{Q1 - Function for MLE optimization}
\end{Program}


\begin{Program}[!htb]
\begin{lstlisting}[style=Matlab-editor,basicstyle=\mlttfamily\footnotesize]
%Q2 & Q4 - Function: NCTPDF ------------------------------------------
function pdfln = Q2_nctpdf ( x , nu , gam)
vn2 = (nu + 1) / 2; rho = x .^2;
pdfln = gammaln( vn2 ) - 1/2*log( pi*nu) - gammaln( nu / 2 ) - vn2*log1p ( rho / nu) ;
if ( all (gam == 0) ) , return , end
idx = ( pdfln >= -37) ; 
if (any( idx ))
gcg = gam.^2 ; pdfln = pdfln - 0.5*gcg ; xcg = x .*gam;
term = 0.5*log(2) + log (xcg) - 0.5*log(max( realmin , nu+rho ) ) ;
term ( term == -inf ) = log ( realmin ) ; term ( term == + inf ) = log ( realmax ) ;
maxiter = 1e4 ; k = 0;
logterms = gammaln ((nu+1+k) / 2 ) - gammaln( k+1) - gammaln( vn2 ) + k*term ;
fractions = real (exp( logterms ) ) ; logsumk = log ( fractions ) ;
while ( k < maxiter)
k = k + 1;
logterms = gammaln ( ( nu+1+k ) / 2 ) - gammaln( k+1) - gammaln( vn2 ) + k*term( idx ) ;
fractions = real (exp( logterms-logsumk(idx))) ;
logsumk( idx ) = logsumk( idx ) + log1p ( fractions ) ;
idx(idx) = (abs(fractions) > 1e-4) ; if ( all ( idx == false ) ) , break , end
end
pdfln = real ( pdfln+logsumk) ;
end
\end{lstlisting}
\caption{Q2 - Q4 - Function NCTPDF}
\label{Q2 & Q4 - Function NCTPDF}
\end{Program}

\begin{Program}[!htb]
\begin{lstlisting}[style=Matlab-editor,basicstyle=\mlttfamily\footnotesize] 
%Q3 - Function: AsymstableES --------------------------------------------
function [ES, VaR] = asymstableES (xi, a, b, mu, scale)
if nargin < 3 , b=0; end , if nargin < 4 , mu=0; end
if nargin < 5 , scale = 1; end 

% Get q, the quantile from the S(0 ,1) distribution
opt=optimset('Display', 'off', 'TolX', 1e-6) ;
q=fzero(@(x)stabcdfroot(x,xi,a,b), -6); VaR=mu+scale*q;

if (q == 0)
 t0 = (1 / a) * atan ( b * tan ( pi * a /2) ) ;
 ES = ((2 * gamma((a-1)/a)) / (pi - 2*t0)) * (cos(t0) / cos(a*t0)^(1/a));
 return;
end

ES=(scale*Stoy(q, a, b) / xi)+mu;

end

function diff = stabcdfroot(x, xi, a, b)
[~,F] = asymstab(x, a, b);
diff = F-xi;
end
\end{lstlisting}
\caption{Q3 - Function AsymstableES}
\label{Q3 - Function AsymstableES}
\end{Program}

\end{document}
